%%=============================================================================
%% Conclusie
%%=============================================================================

\chapter{Conclusie}%
\label{ch:conclusie}

% TO DO: Trek een duidelijke conclusie, in de vorm van een antwoord op de
% onderzoeksvra(a)g(en). Wat was jouw bijdrage aan het onderzoeksdomein en
% hoe biedt dit meerwaarde aan het vakgebied/doelgroep? 
% Reflecteer kritisch over het resultaat. In Engelse teksten wordt deze sectie
% ``Discussion'' genoemd. Had je deze uitkomst verwacht? Zijn er zaken die nog
% niet duidelijk zijn?
% Heeft het onderzoek geleid tot nieuwe vragen die uitnodigen tot verder 
%onderzoek?

Het uitgangspunt van deze bachelorproef was om te onderzoeken of we tot een oplossing konden komen voor de installatie van Hadoop, Spark en Kafka op basis van container technologie, met extra aandacht voor security en stabiliteit, ter ondersteuning van de oefeningen en examens aan HOGENT.
\newline
De mogelijke container technologieën waren onder andere Docker en Kubernetes.
\newline
\newline
Het zijn de aandachtspunten security en stabiliteit, en het mogelijke gebruik van Kubernetes, die in de richting van een oplossing met gescheiden omgevingen hebben geleid. 
\newline
Gezien de complexiteit van het opzetten van de Big Data oplossingen in een gedeelde cluster, zou onze oplossing een aantal zaken kunnen
vereenvoudigen om een haalbare configuratie te realiseren.
\newline
\newline
Met het gebruik van een lokale Docker Desktop en Kubernetes omgeving wordt aangetoond dat container technologie een mogelijke en interessante piste is voor dit soort installaties.
\newline
De uitgewerkte oplossing blijkt te veeleisend te zijn voor de beschikbare resources op het VIC. Wat uitgewerkt is, is eerder geschikt voor een publieke cloud omgeving waar je zo goed als ongelimiteerd nodes kunt starten en stoppen in functie van de behoeften en waar op het moment dat er geen resources gebruikt worden de nodes uitstaan (inclusief dat er dus ook geen kosten aan verbonden zijn).
\newline
\newline
Voor installatie op het VIC moet er dus verder onderzoek gebeuren en rekening houdende met het feit dat er nog geen Kubernetes is geïnstalleerd. Docker Swarm is dan een voor de handliggende optie welke eenvoudiger zal moeten zijn in gebruik en minder resources gebruikt dan Kubernetes.
\newline
De oplossing zou dan een gedeelde cluster zijn, en voor het security aspect moet in eerste instantie best gekeken worden naar Apache Ranger.
\newline
De grootste zorg bij een gedeelde cluster blijft het risico, zoals eerder vermeld, op de ``lawaaierige buur'', hiervoor hebben we volgende ideeën:
\newline
\begin{itemize}
    \item Voldoende testen voorzien om de invloed van een zeer belastende query of programma op de andere gebruikers te meten.
    \item Aangezien de verwerking vooral in Spark worker nodes gebeurt, het eenvoudig maken om snel extra nodes te kunnen toevoegen.
\end{itemize}

