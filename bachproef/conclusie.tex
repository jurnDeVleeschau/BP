%%=============================================================================
%% Conclusie
%%=============================================================================

\chapter{Conclusie}%
\label{ch:conclusie}

% TO DO: Trek een duidelijke conclusie, in de vorm van een antwoord op de
% onderzoeksvra(a)g(en). Wat was jouw bijdrage aan het onderzoeksdomein en
% hoe biedt dit meerwaarde aan het vakgebied/doelgroep? 
% Reflecteer kritisch over het resultaat. In Engelse teksten wordt deze sectie
% ``Discussion'' genoemd. Had je deze uitkomst verwacht? Zijn er zaken die nog
% niet duidelijk zijn?
% Heeft het onderzoek geleid tot nieuwe vragen die uitnodigen tot verder 
%onderzoek?

Het uitgangspunt van deze bachelorproef was om te onderzoeken of we tot een oplossing konden komen voor de installatie van Hadoop, Spark en Kafka op basis van containertechnologie, met extra aandacht voor security en stabiliteit, ter ondersteuning van de oefeningen en examens aan HOGENT.
\newline
De mogelijke containertechnologieën waren onder andere Docker en Kubernetes.
\newline
\newline
Het zijn de aandachtspunten security en stabiliteit, en het mogelijke gebruik van Kubernetes, die ons in de richting van gescheiden omgevingen hebben geleid. TODO Gezien de mogelijke complexiteit van het opzetten van de Big Data oplossingen in een gedeelde cluster hoopten we op die manier ook een aantal zaken te kunnen vereenvoudigen om binnen de beperkte tijd van de bachelorproef een haalbare configuratie te bedenken en realiseren.
\newline
\newline
Door gebruik te maken van een lokale Docker Desktop en Kubernetes omgeving hebben we aangetoond dat containertechnologie een mogelijke en interessante piste is voor dit soort installaties.
\newline
Uiteindelijk bleek wel dat deze oplossing te veeleisend is voor de beschikbare resources op het VIC. De aanpak die we voor ogen hadden, en die we uitgewerkt hebben, zou eerder geschikt zijn voor een publieke cloud omgeving waar je zo goed als ongelimiteerd nodes kunt starten en stoppen, naargelang je behoeften, en waar er geen resources gebruikt worden (en er dus ook geen kosten zijn) op momenten dat de nodes niet nodig zijn en uit staan.
\newline
\newline
Voor installatie op het VIC moet er dus verder onderzoek gebeuren en rekening houdende met het feit dat er nog geen Kubernetes is geïnstalleerd en dat Docker Swarm eenvoudiger zou moeten zijn in gebruik en dat het minder resources gebruikt dan Kubernetes is dat de voor de hand liggende optie.
\newline
De oplossing zou dan een gedeelde cluster zijn, en voor het security aspect moet in eerste instantie best gekeken worden naar Apache Ranger.
\newline
Onze grootste zorg bij een gedeelde cluster blijft het risico, zoals eerder vermeld, op de ``lawaaierige buur'', hiervoor hebben we volgende ideeën:
\newline
\begin{itemize}
    \item Voldoende testen voorzien om de invloed van een zeer belastende query of programma op de andere gebruikers te meten.
    \item Aangezien de verwerking vooral in Spark worker nodes gebeurt, het eenvoudig maken om snel extra nodes te kunnen toevoegen.
\end{itemize}

