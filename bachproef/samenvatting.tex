%%=============================================================================
%% Samenvatting
%%=============================================================================

% TO DO: De "abstract" of samenvatting is een kernachtige (~ 1 blz. voor een
% thesis) synthese van het document.
%
% Een goede abstract biedt een kernachtig antwoord op volgende vragen:
%
% 1. Waarover gaat de bachelorproef?
% 2. Waarom heb je er over geschreven?
% 3. Hoe heb je het onderzoek uitgevoerd?
% 4. Wat waren de resultaten? Wat blijkt uit je onderzoek?
% 5. Wat betekenen je resultaten? Wat is de relevantie voor het werkveld?
%
% Daarom bestaat een abstract uit volgende componenten:
%
% - inleiding + kaderen thema
% - probleemstelling
% - (centrale) onderzoeksvraag
% - onderzoeksdoelstelling
% - methodologie
% - resultaten (beperk tot de belangrijkste, relevant voor de onderzoeksvraag)
% - conclusies, aanbevelingen, beperkingen
%
% LET OP! Een samenvatting is GEEN voorwoord!

%%---------- Nederlandse samenvatting -----------------------------------------
%
% TO DO: Als je je bachelorproef in het Engels schrijft, moet je eerst een
% Nederlandse samenvatting invoegen. Haal daarvoor onderstaande code uit
% commentaar.
% Wie zijn bachelorproef in het Nederlands schrijft, kan dit negeren, de inhoud
% wordt niet in het document ingevoegd.

% \IfLanguageName{english}{%
% \selectlanguage{dutch}
% \chapter*{Samenvatting}
% \selectlanguage{english}
% }{}

%%---------- Samenvatting -----------------------------------------------------
% De samenvatting in de hoofdtaal van het document

\chapter*{\IfLanguageName{dutch}{Samenvatting}{Abstract}}

In het kader van de opleiding Toegepaste Informatica HOGENT worden de Big Data frameworks Hadoop, Spark en Kafka lokaal op de laptop van de student geïnstalleerd voor de oefeningen.
De doelstelling van deze bachelorproef is uit te zoeken of containertechnologie kan gebruikt worden om die gecombineerde installaties van Big Data frameworks te automatiseren, met een focus op efficiënt gebruik van resources, security en stabiliteit, schaalbaarheid en logging. De bedoeling is dat de resultaten van dit onderzoek in de volgende jaren kunnen gebruikt worden om de lessen van het vak ``Big Data Processing'' te faciliteren.
\newline
Hiervoor werden de installatie en configuratie van deze frameworks bestudeerd om tot een werkende oplossing te komen, en werd containertechnologie onderzocht, Docker (Compose en Swarm) en Kubernetes, die kan gebruikt worden om de software centraal te installeren, en niet langer op de laptop van de student.
\newline
De aandachtspunten security en stabiliteit hebben ons in de richting van strict gescheiden omgevingen voor elke student geleidt, in plaats van ëën gedeelde omgeving voor alle studenten. Ook omdat we hoopten op die manier een aantal zaken te kunnen vereenvoudigen om binnen de beperkte tijd van de bachelorproef een haalbare configuratie te realiseren.
\newline
Tijdens het onderzoek hebben TODO we veel geleerd over het gebruik van containers, en door gebruik te maken van een lokale Docker Desktop en Kubernetes omgeving zijn we tot een werkende omgeving op Kubernetes gekomen en hebben TODO we aangetoond dat containertechnologie een mogelijke en interessante piste is voor dit soort installaties.
\newline
Uiteindelijk bleek wel dat deze oplossing te veeleisend is voor de beschikbare resources op het VIC. Verder onderzoek is dus nodig naar de gedeelde omgeving als aanpak, waarbij Docker Swarm kan onderzocht worden als minder-belastend alternatief voor Kubernetes. In de gedeelde omgeving moet er ook meer aandacht zijn voor de security en stabiliteit, hiervoor doen we een aantal aanbevelingen in de conclusie.
\newline
\newline
