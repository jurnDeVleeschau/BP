%%=============================================================================
%% Inleiding
%%=============================================================================

\chapter{\IfLanguageName{dutch}{Inleiding}{Introduction}}%
\label{ch:inleiding}

%De inleiding moet de lezer net genoeg informatie verschaffen om het onderwerp te begrijpen en in te zien waarom de onderzoeksvraag de moeite waard is om te onderzoeken. In de inleiding ga je literatuurverwijzingen beperken, zodat de tekst vlot leesbaar blijft. Je kan de inleiding verder onderverdelen in secties als dit de tekst verduidelijkt. Zaken die aan bod kunnen komen in de inleiding~\autocite{Pollefliet2011}:
%\begin{itemize}
%  \item context, achtergrond
%  \item afbakenen van het onderwerp
%  \item verantwoording van het onderwerp, methodologie
%  \item probleemstelling
%  \item onderzoeksdoelstelling
%  \item onderzoeksvraag
%  \item \ldots
%\end{itemize}
In het kader van de opleiding Toegepaste Informatica HOGENT dient de student specifieke software pakketten te installeren, die gebruikt worden tijdens de les, de taken en de examens.
De doelstelling is uit te zoeken of en op welke wijze container technologie kan gebruikt worden om dit zo efficiënt mogelijk te laten verlopen voor alle betrokkenen. Specifiek zouden de resultaten van dit onderzoek in de volgende jaren toegepast kunnen worden om de lessen van het vak ``Big Data Processing'' te faciliteren.

\section{\IfLanguageName{dutch}{Probleemstelling}{Problem Statement}}%
\label{sec:probleemstelling}
Tijdens de opleiding aan HOGENT worden de Big Data frameworks Hadoop, Spark en Kafka gebruikt. Op dit moment gaat er kostbare lestijd verloren bij het installeren van alle software op de laptop van de student. Om dit efficiënt te laten verlopen onderzoeken we of container technologie kan worden gebruikt om de verschillende applicaties te installeren, rekening houdend met vereisten als efficiënt gebruik van hulpbronnen, beveiliging en stabiliteit (gebruikers van elkaar afschermen), schaalbaarheid en logging.

\section{\IfLanguageName{dutch}{Onderzoeksvraag}{Research question}}%
\label{sec:onderzoeksvraag}
Dit onderzoek focust op het realiseren van een installatie van Hadoop, Spark en Kafka ter ondersteuning van de lessen Big Data Processing en daaruit volgt volgende onderzoeksvraag:
\textbf{Kan container technologie gebruikt worden om de gecombineerde installatie van Hadoop, Spark en Kafka te automatiseren ?}

\section{\IfLanguageName{dutch}{Onderzoeksdoelstelling}{Research objective}}%
\label{sec:onderzoeksdoelstelling}

%Wat is het beoogde resultaat van je bachelorproef? Wat zijn de criteria voor succes? Beschrijf die zo concreet mogelijk. Gaat het bv.\ om een proof-of-concept, een prototype, een verslag met aanbevelingen, een vergelijkende studie, enz.
Het beoogde resultaat van het onderzoek is meer inzicht te verschaffen in het gebruik van container technologie, toegepast op  installaties van Hadoop, Spark en Kafka en te komen tot aanbevelingen voor de installatie op het VIC\footnote{HOGENT Virtual IT Company}. Daarbij worden een aantal proof-of-concepts uitgevoerd om meer inzicht te krijgen in de complexiteit hiervan.

\section{\IfLanguageName{dutch}{Opzet van deze bachelorproef}{Structure of this bachelor thesis}}%
\label{sec:opzet-bachelorproef}

De rest van deze bachelorproef is als volgt opgebouwd:

In Hoofdstuk~\ref{ch:stand-van-zaken} geven we een overzicht van de stand van zaken binnen het onderzoeksdomein, op basis van een literatuurstudie.

In Hoofdstuk~\ref{ch:methodologie} gaan we dieper in op wat we geleerd hebben in Hoofdstuk~\ref{ch:stand-van-zaken} en beschrijven we hoe we tot bepaalde beslissingen zijn gekomen, ondersteund door een aantal proof-of-concepts. De stappen nodig om een antwoord te kunnen formuleren op de onderzoeksvragen worden toegelicht.

% TO DO: Vul hier aan voor je eigen hoofstukken, één of twee zinnen per hoofdstuk

In Hoofdstuk~\ref{ch:conclusie}, tenslotte, wordt de conclusie gegeven en een antwoord geformuleerd op de onderzoeksvragen. Daarbij wordt ook een aanzet gegeven voor toekomstig onderzoek binnen dit domein.
